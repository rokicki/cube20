\input cwebmac
\def\mod{\mathop{mod}}


\N{1}{1}Introduction.
This is a solver for Rubik's Cube, using Herbert Kociemba's
two-phase algorithm.

\Y\B\4\X1:\.{twophase.cpp }\X${}\E{}$\6
\&{const} \&{char} ${}{*}\.{BANNER}\K\.{"This\ is\ twophase\ 1.}\)\.{1,\ (C)\
2010-2012\ Tom}\)\.{as\ Rokicki.\ \ All\ Rig}\)\.{hts\ Reserved."};{}$\6
\8\#\&{include} \.{"phase1prune.h"}\6
\8\#\&{include} \.{"phase2prune.h"}\6
\8\#\&{include} \.{<pthread.h>}\6
\8\#\&{include} \.{<iostream>}\6
\8\#\&{include} \.{<map>}\6
\8\#\&{include} \.{<cstdio>}\6
\&{using} \&{namespace} \&{std};\7
\X2:Data declarations\X\6
\X10:Utility functions\X\6
\X11:Solver class\X\7
\&{int} \\{main}(\&{int} \\{argc}${},\39{}$\&{char} ${}{*}\\{argv}[\,]){}$\1\1%
\2\2\6
${}\{{}$\1\6
\&{double} \\{progstart}${}\K\\{walltime}(\,);{}$\7
\\{duration}(\,);\6
\X3:Parse arguments\X\6
\X6:Initialize the program\X\6
\X34:Handle the work\X\6
\X30:Print summary\X\6
\4${}\}{}$\2\par
\fi

\M{2}There is actually very little cube code in this program.  Almost all
the work is handled by the code in {\tt cubepos}, {\tt kocsymm},
{\tt phase1prune}, and {\tt phase2prune}; the code in this file is
predominantly bookkeeping, thread management, and the like.  The first
thing we take up is argument parsing.  Two arguments we know we
need up front include a verbosity level (the default is 1, but the
\PB{${-}\|q$} option makes it 0 and the \PB{${-}\|v$} option makes it 2), and a
thread count.

\Y\B\4\X2:Data declarations\X${}\E{}$\6
\&{int} \\{verbose}${}\K\T{1};{}$\6
\&{int} \\{numthreads}${}\K\T{1};{}$\6
\&{const} \&{int} \.{MAX\_THREADS}${}\K\T{32}{}$;\par
\As4, 7, 15, 19\ETs28.
\U1.\fi

\M{3}Parsing the arguments is boilerplate code.

\Y\B\4\X3:Parse arguments\X${}\E{}$\6
\&{while} ${}(\\{argc}>\T{1}\W\\{argv}[\T{1}][\T{0}]\E\.{'-'}){}$\5
${}\{{}$\1\6
${}\\{argc}\MM;{}$\6
${}\\{argv}\PP;{}$\6
\&{switch} (\\{argv}[\T{0}][\T{1}])\5
${}\{{}$\1\6
\4\&{case} \.{'v'}:\5
${}\\{verbose}\PP;{}$\6
\&{break};\6
\4\&{case} \.{'q'}:\5
${}\\{verbose}\K\T{0};{}$\6
\&{break};\6
\4\&{case} \.{'t'}:\6
\&{if} ${}(\\{sscanf}(\\{argv}[\T{1}],\39\.{"\%d"},\39{\AND}\\{numthreads})\I%
\T{1}){}$\1\6
\&{error} (\.{"!\ bad\ thread\ count\ }\)\.{argument"})\1\5
;\2\2\6
\&{if} ${}(\\{numthreads}<\T{1}\V\\{numthreads}>\.{MAX\_THREADS}){}$\1\6
\&{error} (\.{"!\ bad\ value\ for\ thr}\)\.{ead\ count"})\1\5
;\2\2\6
${}\\{argc}\MM;{}$\6
${}\\{argv}\PP;{}$\6
\&{break};\6
\X5:More arguments\X\6
\4\&{default}:\6
\&{error} (\.{"!\ bad\ argument"})\1\5
;\2\6
\4${}\}{}$\2\6
\4${}\}{}$\2\par
\U1.\fi

\M{4}The two-phase algorithm discovers solutions for each position of
decreasing distance.  Nominally, it runs forever for any single
position, always seeking a shorter position.  For this command-line
program there are two ways to limit its execution.  The first is to
limit the count of phase two probes that it will do per position.  The
next is to give a target solution length, such that if a solution of
that length or shorter is found, search will end.  We also track
how many total phase 2 probes there were.

\Y\B\4\X2:Data declarations\X${}\mathrel+\E{}$\6
\&{int} \\{target\_length}${}\K\T{0};{}$\6
\&{long} \&{long} \\{phase2limit}${}\K\T{\^ffffffffffffff\$L\$L};{}$\6
\&{long} \&{long} \\{phase2total}${}\K\T{0\$L\$L}{}$;\par
\fi

\M{5}Both of these can be set from the command line.

\Y\B\4\X5:More arguments\X${}\E{}$\6
\4\&{case} \.{'M'}:\6
\&{if} ${}(\\{sscanf}(\\{argv}[\T{1}],\39\.{"\%lld"},\39{\AND}\\{phase2limit})%
\I\T{1}){}$\1\6
\&{error} (\.{"!\ bad\ argument\ to\ -}\)\.{M"})\1\5
;\2\2\6
${}\\{argc}\MM;{}$\6
${}\\{argv}\PP;{}$\6
\&{break};\6
\4\&{case} \.{'s'}:\6
\&{if} ${}(\\{sscanf}(\\{argv}[\T{1}],\39\.{"\%d"},\39{\AND}\\{target\_length})%
\I\T{1}){}$\1\6
\&{error} (\.{"!\ bad\ argument\ to\ -}\)\.{s"})\1\5
;\2\2\6
\&{if} ${}(\\{target\_length}\G\.{MAX\_MOVES}){}$\1\5
${}\\{target\_length}\K\.{MAX\_MOVES}-\T{1};{}$\2\6
${}\\{argc}\MM;{}$\6
${}\\{argv}\PP;{}$\6
\&{break};\par
\As8\ET20.
\U3.\fi

\M{6}If neither of these two arguments are supplied, and we're
not running in verbose mode of 2 or greater, that's an
error because no output will be generated.

\Y\B\4\X6:Initialize the program\X${}\E{}$\6
\&{if} ${}(\\{phase2limit}\G\T{\^ffffffffffffff\$L\$L}\W\\{target\_length}\E%
\T{0}\W\\{verbose}\Z\T{1}){}$\1\6
\&{error} (\.{"!\ must\ specify\ -M,\ }\)\.{-s,\ or\ -v"})\1\5
;\2\2\par
\As9\ET16.
\U1.\fi

\M{7}Usually the pruning tables are read from disk; if they don't exist,
they are created, and then written to disk.  If for some reason
you do not want to write the pruning tables to disk, you can use
the \PB{${-}\|W$} option to inhibit this.

\Y\B\4\X2:Data declarations\X${}\mathrel+\E{}$\6
\&{int} \\{skipwrite}${}\K\T{0}{}$;\par
\fi

\M{8}Parsing this argument is easy.

\Y\B\4\X5:More arguments\X${}\mathrel+\E{}$\6
\4\&{case} \.{'W'}:\5
${}\\{skipwrite}\PP;{}$\6
\&{break};\par
\fi

\M{9}If we are not running in quiet mode, we always print the banner as
the first part of initialization, before we load the pruning tables
since those can be slow.  We use both a phase 1 and a phase 2 pruning
table, so we initialize these in turn.

\Y\B\4\X6:Initialize the program\X${}\mathrel+\E{}$\6
\&{if} (\\{verbose})\1\5
${}\\{cout}\LL\.{BANNER}\LL\\{endl}\LL\\{flush};{}$\2\6
\\{phase1prune}\DC\\{init}(\\{skipwrite});\6
\\{phase2prune}\DC\\{init}(\\{skipwrite});\par
\fi

\M{10}We know that no solution will ever be longer than 32 moves.
The maximum distance from phase 1 is 12, and from phase 2 is 18.
We give it two extra just for good measure.

\Y\B\4\X10:Utility functions\X${}\E{}$\6
\&{const} \&{int} \.{MAX\_MOVES}${}\K\T{32}{}$;\par
\As17, 22, 27, 29\ETs32.
\U1.\fi

\N{1}{11}Multithreading.
This program is intended to batch-solve many positions, rather than
just try to solve a particular position; for this reason, we use
threads and assign each position a thread (up to the maximum number of
threads specified by the user).  We do not solve a single position by
multiple threads, because of the complexity of managing communication
between the threads.  Thus, the solution work and all solution state is
maintained in a worker thread called a \PB{\\{twophasesolver}}.  Each instance
of this class is small, so we preallocate all needed instances.  We
introduce a bit of padding so multiple threads don't fight over the
same cache lines.

\Y\B\4\X11:Solver class\X${}\E{}$\6
\&{class} \&{twophasesolver} ${}\{{}$\1\6
\4\&{public}:\6
\&{twophasesolver}(\,)\1\1\2\2\6
${}\{\,\}{}$\7
\&{cubepos} \\{pos};\SHC{ position to solve }\6
\&{long} \&{long} \\{phase2probes};\6
\&{int} \\{bestsol};\SHC{ length of the best solution }\6
\&{int} \\{finished};\SHC{ set to true to terminate }\6
\&{int} \\{curm};\SHC{ what orientation we are working on }\6
\&{int} \\{solmap};\SHC{ the orientation the solution is in }\6
\&{int} \\{seq};\SHC{ a serial number for this position }\7
\X12:Solver data\X\6
\X13:Solver methods\X\7
\&{char} \\{pad}[\T{256}];\2\6
${}\}{}$ \\{solvers}[\.{MAX\_THREADS}];\par
\U1.\fi

\M{12}As we perform the searches, we place the moves we have needed into
the following array.

\Y\B\4\X12:Solver data\X${}\E{}$\6
\&{unsigned} \&{char} \\{moves}[\.{MAX\_MOVES}];\6
\&{unsigned} \&{char} \\{bestmoves}[\.{MAX\_MOVES}];\par
\A18.
\U11.\fi

\M{13}Our basic method for solving is here.  This initializes the
structure and starts the solution process.  The result is stored
in the fields of the solver structure.

\Y\B\4\X13:Solver methods\X${}\E{}$\6
\&{void} \\{solve}(\&{int} \\{seqarg}${},\39{}$\&{cubepos} ${}{\AND}\\{cp}){}$%
\1\1\2\2\6
${}\{{}$\1\6
\X14:Initialize the solver\X\6
\X21:Set up position invariants\X\6
\X23:Solve one position\X\6
\X26:Check and report solution\X\6
\4${}\}{}$\2\par
\As24, 25, 31\ETs33.
\U11.\fi

\M{14}We initialize the fields we have declared so far.

\Y\B\4\X14:Initialize the solver\X${}\E{}$\6
$\\{pos}\K\\{cp};{}$\6
${}\\{phase2probes}\K\T{0};{}$\6
${}\\{bestsol}\K\.{MAX\_MOVES}{}$;\SHC{ no solution found }\6
${}\\{finished}\K\T{0};{}$\6
${}\\{seq}\K\\{seqarg}{}$;\par
\U13.\fi

\M{15}Access to the output stream, or any other shared state, requires
a global mutex.

\Y\B\4\X2:Data declarations\X${}\mathrel+\E{}$\6
\\{pthread\_mutex\_t}\\{my\_mutex};\par
\fi

\M{16}We initialize the mutex.

\Y\B\4\X6:Initialize the program\X${}\mathrel+\E{}$\6
$\\{pthread\_mutex\_init}({\AND}\\{my\_mutex},\39\NULL){}$;\par
\fi

\M{17}We call these methods to acquire and release the mutex.

\Y\B\4\X10:Utility functions\X${}\mathrel+\E{}$\6
\&{void} \\{get\_global\_lock}(\,)\1\1\2\2\6
${}\{{}$\1\6
${}\\{pthread\_mutex\_lock}({\AND}\\{my\_mutex});{}$\6
\4${}\}{}$\2\7
\&{void} \\{release\_global\_lock}(\,)\1\1\2\2\6
${}\{{}$\1\6
${}\\{pthread\_mutex\_unlock}({\AND}\\{my\_mutex});{}$\6
\4${}\}{}$\2\par
\fi

\M{18}This program uses a new, six-axis technique, to find a solution as
quickly as possible.  Normally the two-phase method just repeats phase
1 and phase 2 for increasing depths of phase 1 from the given input
position.  Some positions, however, are recalcitrant, and do not have
a nice short phase 1 solution leading to a short phase 2 solution.  To
make it more likely to find a solution quickly, we consider all three
axis reorientations.  Further, we also consider the inverse position
with all three axis reorientations, for a grand total of six different
starting positions.  For positions with symmetry or antisymmetry, we
do not want to cover the same ground multiple times, so we need to
calculate which of these six combinations are distinct.

For each of the six starting positions, we construct a \PB{\&{kocsymm}} and a
\PB{\&{cubepos}} to carry the state, calculate the minimum phase one depth,
and check which are unique.  We need to declare these fields.

\Y\B\4\X12:Solver data\X${}\mathrel+\E{}$\6
\&{kocsymm} \\{kc6}[\T{6}]${},{}$ \\{kccanon6}[\T{6}];\6
\&{cubepos} \\{cp6}[\T{6}];\6
\&{permcube} \\{pc6}[\T{6}];\6
\&{int} \\{mindepth}[\T{6}];\6
\&{char} \\{uniq}[\T{6}];\6
\&{int} \\{minmindepth};\par
\fi

\M{19}Sometimes we want to run the twophase algorithm only working
with a specific subset of the possible axes.  This mask gives
that information.

\Y\B\4\X2:Data declarations\X${}\mathrel+\E{}$\6
\&{int} \\{axesmask}${}\K\T{63}{}$;\par
\fi

\M{20}We use the -a option to set this.

\Y\B\4\X5:More arguments\X${}\mathrel+\E{}$\6
\4\&{case} \.{'a'}:\5
${}\\{axesmask}\K\\{atol}(\\{argv}[\T{1}]);{}$\6
${}\\{argv}\PP;{}$\6
${}\\{argc}\MM;{}$\6
\&{break};\par
\fi

\M{21}Initializing these is fairly easy.  We need to know what
reorientations change the axis; from \PB{\&{cubepos}} we know that each group
of sixteen consecutive remappings maintains the up/down axis, so we
use reorientations 0, 16, and 32.  When comparing positions for
equality, we want to use the 16-way canonicalization that preserves
the up/down axis; the \PB{\&{kocsymm}} canonicalization preserves this, so
we use this to preface a more involved comparison.

\Y\B\4\X21:Set up position invariants\X${}\E{}$\6
$\\{minmindepth}\K\.{MAX\_MOVES};{}$\7
\&{cubepos} \\{cpi}${},{}$ \\{cp2};\7
${}\\{pos}.\\{invert\_into}(\\{cpi});{}$\7
\&{int} \\{ind}${}\K\T{0};{}$\7
\&{for} (\&{int} \\{inv}${}\K\T{0};{}$ ${}\\{inv}<\T{2};{}$ ${}\\{inv}\PP){}$\1%
\6
\&{for} (\&{int} \\{mm}${}\K\T{0};{}$ ${}\\{mm}<\T{3};{}$ ${}\\{mm}\PP,\39%
\\{ind}\PP){}$\5
${}\{{}$\1\6
\&{int} \|m${}\K\.{KOCSYMM}*\\{mm};{}$\7
\&{if} (\\{inv})\1\5
${}\\{cpi}.\\{remap\_into}(\|m,\39\\{cp2});{}$\2\6
\&{else}\1\5
${}\\{pos}.\\{remap\_into}(\|m,\39\\{cp2});{}$\2\6
${}\\{cp6}[\\{ind}]\K\\{cp2};{}$\6
${}\\{kc6}[\\{ind}]\K\&{kocsymm}(\\{cp2});{}$\6
${}\\{pc6}[\\{ind}]\K\&{permcube}(\\{cp2});{}$\6
${}\\{kc6}[\\{ind}].\\{canon\_into}(\\{kccanon6}[\\{ind}]);{}$\6
${}\\{mindepth}[\\{ind}]\K\\{phase1prune}\DC\\{lookup}(\\{kc6}[\\{ind}]);{}$\6
\&{if} ${}(\\{mindepth}[\\{ind}]<\\{minmindepth}){}$\1\5
${}\\{minmindepth}\K\\{mindepth}[\\{ind}];{}$\2\6
${}\\{uniq}[\\{ind}]\K\T{1}\AND(\\{axesmask}\GG\\{ind});{}$\6
\&{for} (\&{int} \|i${}\K\T{0};{}$ ${}\|i<\\{ind};{}$ ${}\|i\PP){}$\1\6
\&{if} ${}(\\{uniq}[\|i]\W\\{kccanon6}[\\{ind}]\E\\{kccanon6}[\|i]\W\\{sloweq}(%
\\{cp6}[\\{ind}],\39\\{cp6}[\|i])){}$\5
${}\{{}$\1\6
${}\\{uniq}[\\{ind}]\K\T{0};{}$\6
\&{break};\6
\4${}\}{}$\2\2\6
\&{if} ${}(\\{verbose}>\T{1}){}$\5
${}\{{}$\1\6
\\{get\_global\_lock}(\,);\6
${}\\{cout}\LL\.{"Axis\ "}\LL\\{ind}\LL\.{"\ depth\ "}\LL\\{mindepth}[\\{ind}]%
\LL\.{"\ uniq\ "}\LL{}$(\&{int}) \\{uniq}[\\{ind}]${}\LL\\{endl};{}$\6
\\{release\_global\_lock}(\,);\6
\4${}\}{}$\2\6
\4${}\}{}$\2\2\par
\U13.\fi

\M{22}We need a utility method that does a slow comparison between two
\PB{\&{cubepos}} and returns true if there's any reorientation, preserving the
up/down axis, of one that is the same as the other.

\Y\B\4\X10:Utility functions\X${}\mathrel+\E{}$\6
\&{int} \\{sloweq}(\&{const} \&{cubepos} ${}{\AND}\\{cp1},\39{}$\&{const} %
\&{cubepos} ${}{\AND}\\{cp2}){}$\1\1\2\2\6
${}\{{}$\1\6
\&{cubepos} \\{cp3};\7
\&{for} (\&{int} \|m${}\K\T{0};{}$ ${}\|m<\.{KOCSYMM};{}$ ${}\|m\PP){}$\5
${}\{{}$\1\6
${}\\{cp2}.\\{remap\_into}(\|m,\39\\{cp3});{}$\6
\&{if} ${}(\\{cp1}\E\\{cp3}){}$\1\5
\&{return} \T{1};\2\6
\4${}\}{}$\2\6
\&{return} \T{0};\6
\4${}\}{}$\2\par
\fi

\M{23}Once we have a minimum depth, we can start solving.

\Y\B\4\X23:Solve one position\X${}\E{}$\6
\&{for} (\&{int} \|d${}\K\\{minmindepth};{}$ ${}\|d<\\{bestsol}\W\R%
\\{finished};{}$ ${}\|d\PP){}$\5
${}\{{}$\1\6
\&{for} ${}(\\{curm}\K\T{0};{}$ ${}\\{curm}<\T{6};{}$ ${}\\{curm}\PP){}$\1\6
\&{if} ${}(\\{uniq}[\\{curm}]\W\|d<\\{bestsol}\W\R\\{finished}\W\|d\G%
\\{mindepth}[\\{curm}]){}$\5
${}\{{}$\1\6
\&{if} ${}(\\{verbose}>\T{1}){}$\5
${}\{{}$\1\6
\\{get\_global\_lock}(\,);\6
${}\\{cout}\LL\.{"Orientation\ "}\LL\\{curm}\LL\.{"\ at\ depth\ "}\LL\|d\LL%
\\{endl};{}$\6
\\{release\_global\_lock}(\,);\6
\4${}\}{}$\2\6
${}\\{solvep1}(\\{kc6}[\\{curm}],\39\\{pc6}[\\{curm}],\39\|d,\39\T{0},\39%
\.{ALLMOVEMASK},\39\.{CANONSEQSTART});{}$\6
\4${}\}{}$\2\2\6
\4${}\}{}$\2\par
\U13.\fi

\M{24}The phase one solver.

\Y\B\4\X13:Solver methods\X${}\mathrel+\E{}$\6
\&{void} \\{solvep1}(\&{const} \&{kocsymm} ${}{\AND}\\{kc},\39{}$\&{const} %
\&{permcube} ${}{\AND}\\{pc},\39{}$\&{int} \\{togo}${},\39{}$\&{int} %
\\{sofar}${},\39{}$\&{int} \\{movemask}${},\39{}$\&{int} \\{canon})\1\1\2\2\6
${}\{{}$\1\6
\&{if} ${}(\\{togo}\E\T{0}){}$\5
${}\{{}$\1\6
\&{if} ${}(\\{kc}\E\\{identity\_kc}){}$\1\5
${}\\{solvep2}(\\{pc},\39\\{sofar});{}$\2\6
\&{return};\6
\4${}\}{}$\2\6
\&{if} (\\{finished})\1\5
\&{return};\2\6
${}\\{togo}\MM;{}$\7
\&{kocsymm} \\{kc2};\6
\&{permcube} \\{pc2};\6
\&{int} \\{newmovemask};\7
\&{while} ${}(\R\\{finished}\W\\{movemask}){}$\5
${}\{{}$\1\6
\&{int} \\{mv}${}\K\\{ffs}(\\{movemask})-\T{1};{}$\7
${}\\{movemask}\MRL{\AND{\K}}\\{movemask}-\T{1};{}$\6
${}\\{kc2}\K\\{kc};{}$\6
${}\\{kc2}.\\{move}(\\{mv});{}$\7
\&{int} \\{nd}${}\K\\{phase1prune}\DC\\{lookup}(\\{kc2},\39\\{togo},\39%
\\{newmovemask});{}$\7
\&{if} ${}(\\{nd}\Z\\{togo}\W(\\{togo}\E\\{nd}\V\\{togo}+\\{nd}\G\T{5})){}$\5
${}\{{}$\1\6
${}\\{pc2}\K\\{pc};{}$\6
${}\\{pc2}.\\{move}(\\{mv});{}$\6
${}\\{moves}[\\{sofar}]\K\\{mv};{}$\7
\&{int} \\{new\_canon}${}\K\&{cubepos}\DC\\{next\_cs}(\\{canon},\39\\{mv});{}$\7
${}\\{solvep1}(\\{kc2},\39\\{pc2},\39\\{togo},\39\\{sofar}+\T{1},\39%
\\{newmovemask}\AND\&{cubepos}\DC\\{cs\_mask}(\\{new\_canon}),\39\\{new%
\_canon});{}$\6
\4${}\}{}$\2\6
\4${}\}{}$\2\6
\4${}\}{}$\2\par
\fi

\M{25}The phase two code just uses the phase two solver.

\Y\B\4\X13:Solver methods\X${}\mathrel+\E{}$\6
\&{void} \\{solvep2}(\&{const} \&{permcube} ${}{\AND}\\{pc},\39{}$\&{int} %
\\{sofar})\1\1\2\2\6
${}\{{}$\1\6
${}\\{phase2probes}\PP;{}$\7
\&{int} \|d${}\K\\{phase2prune}\DC\\{lookup}(\\{pc});{}$\7
\&{if} ${}(\|d+\\{sofar}<\\{bestsol}){}$\5
${}\{{}$\1\6
\&{moveseq} \\{ms}${}\K\\{phase2prune}\DC\\{solve}(\\{pc},\39\\{bestsol}-%
\\{sofar}-\T{1});{}$\7
\&{if} ${}((\&{int})(\\{ms}.\\{size}(\,))+\\{sofar}<\\{bestsol}\W(\\{ms}.%
\\{size}(\,)>\T{0}\V\\{pc}\E\\{identity\_pc})){}$\5
${}\{{}$\1\6
${}\\{bestsol}\K\\{ms}.\\{size}(\,)+\\{sofar};{}$\6
\&{for} (\&{unsigned} \&{int} \|i${}\K\T{0};{}$ ${}\|i<\\{ms}.\\{size}(\,);{}$
${}\|i\PP){}$\1\5
${}\\{moves}[\\{sofar}+\|i]\K\\{ms}[\|i];{}$\2\6
${}\\{memcpy}(\\{bestmoves},\39\\{moves},\39\\{bestsol});{}$\6
\&{if} ${}(\\{verbose}>\T{1}){}$\5
${}\{{}$\1\6
\\{get\_global\_lock}(\,);\6
${}\\{cout}\LL\.{"New\ solution\ for\ "}\LL\\{seq}\LL\.{"\ at\ "}\LL\\{bestsol}%
\LL\\{endl};{}$\6
\\{release\_global\_lock}(\,);\6
\4${}\}{}$\2\6
${}\\{solmap}\K\\{curm};{}$\6
\&{if} ${}(\\{bestsol}\Z\\{target\_length}){}$\1\5
${}\\{finished}\K\T{1};{}$\2\6
\4${}\}{}$\2\6
\4${}\}{}$\2\6
\&{if} ${}(\\{phase2probes}\G\\{phase2limit}\W\\{bestsol}<\.{MAX\_MOVES}){}$\1\5
${}\\{finished}\K\T{1};{}$\2\6
\4${}\}{}$\2\par
\fi

\M{26}When we have a solution, and we have decided to present it to the user,
we need to first remap it, and then check it.

\Y\B\4\X26:Check and report solution\X${}\E{}$\6
\&{moveseq} \\{sol};\6
\&{int} \|m${}\K\&{cubepos}\DC\\{invm}[(\\{solmap}\MOD\T{3})*\.{KOCSYMM}];{}$\7
\&{for} (\&{int} \|i${}\K\T{0};{}$ ${}\|i<\\{bestsol};{}$ ${}\|i\PP){}$\1\5
${}\\{sol}.\\{push\_back}(\&{cubepos}\DC\\{move\_map}[\|m][\\{bestmoves}[%
\|i]]);{}$\2\6
\&{if} ${}(\\{solmap}\G\T{3}){}$\1\5
${}\\{sol}\K\&{cubepos}\DC\\{invert\_sequence}(\\{sol});{}$\2\7
\&{cubepos} \\{cpt};\7
\&{for} (\&{unsigned} \&{int} \|i${}\K\T{0};{}$ ${}\|i<\\{sol}.\\{size}(\,);{}$
${}\|i\PP){}$\1\5
${}\\{cpt}.\\{move}(\\{sol}[\|i]);{}$\2\6
\&{if} ${}(\\{cpt}\I\\{pos}){}$\1\6
\&{error} (\.{"!\ move\ sequence\ doe}\)\.{sn't\ work"})\1\5
;\2\2\6
${}\\{report}(\\{pos},\39\\{seq},\39\\{phase2probes},\39\\{sol}){}$;\par
\U13.\fi

\M{27}When we display a solution, we give some basic statistics as
well as the solution itself.  If we missed a target, we display
the orignal position in Singmaster format as well.

\Y\B\4\X10:Utility functions\X${}\mathrel+\E{}$\6
\&{void} \\{display}(\&{const} \&{cubepos} ${}{\AND}\\{cp},\39{}$\&{int} %
\\{seq}${},\39{}$\&{long} \&{long} \\{phase2probes}${},\39{}$\&{moveseq} %
\\{sol})\1\1\2\2\6
${}\{{}$\1\6
${}\\{phase2total}\MRL{+{\K}}\\{phase2probes};{}$\6
\&{if} ${}(\\{verbose}\V{}$(\&{int}) \\{sol}${}.\\{size}(\,)>\\{target%
\_length}){}$\5
${}\{{}$\1\6
\&{if} ((\&{int}) \\{sol}${}.\\{size}(\,)>\\{target\_length}){}$\1\5
${}\\{cout}\LL\.{"WARNING:\ missed\ tar}\)\.{get\ for\ "}\LL\\{cp}.%
\\{Singmaster\_string}(\,)\LL\\{endl};{}$\2\6
${}\\{cout}\LL\.{"Solution\ "}\LL\\{seq}\LL\.{"\ len\ "}\LL\\{sol}.\\{size}(\,)%
\LL\.{"\ probes\ "}\LL\\{phase2probes}\LL\\{endl};{}$\6
${}\\{cout}\LL\&{cubepos}\DC\\{moveseq\_string}(\\{sol})\LL\\{endl};{}$\6
\4${}\}{}$\2\6
\4${}\}{}$\2\par
\fi

\N{1}{28}Reporting solutions.
We would like to report solutions in sequential order.  Yet, with
threads, they may have their solutions found in non-sequential order.
To manage this, we keep track of the most recently printed sequence,
and only print a solution if it is the most recent sequence.  If it
is not we queue it up and print it when the missing sequence shows.
We define a structure to hold the relevant information.  We also
define a couple of variables to contain statistics.

\Y\B\4\X2:Data declarations\X${}\mathrel+\E{}$\6
\&{class} \&{solution} ${}\{{}$\1\6
\4\&{public}:\6
\&{solution}(\&{const} \&{cubepos} ${}{\AND}\\{cparg},\39{}$\&{int} %
\\{seqarg}${},\39{}$\&{long} \&{long} \\{p2parg}${},\39{}$\&{moveseq} ${}{\AND}%
\\{solarg}){}$\1\1\2\2\6
${}\{{}$\1\6
${}\\{cp}\K\\{cparg};{}$\6
${}\\{seq}\K\\{seqarg};{}$\6
${}\\{phase2probes}\K\\{p2parg};{}$\6
${}\\{moves}\K\\{solarg};{}$\6
\4${}\}{}$\2\7
\&{solution}(\,)\1\1\2\2\6
${}\{\,\}{}$\7
\&{cubepos} \\{cp};\6
\&{int} \\{seq};\6
\&{long} \&{long} \\{phase2probes};\6
\&{moveseq} \\{moves};\2\6
${}\};{}$\6
${}\&{map}\langle\&{int},\39\&{solution}\rangle{}$ \\{queue};\6
\&{int} \\{next\_sequence}${}\K\T{1};{}$\6
\&{int} \\{missed\_target}${}\K\T{0};{}$\6
\&{int} \\{solved}${}\K\T{0}{}$;\par
\fi

\M{29}Our reporting function does the main work, with the global lock.

\Y\B\4\X10:Utility functions\X${}\mathrel+\E{}$\6
\&{void} \\{report}(\&{const} \&{cubepos} ${}{\AND}\\{cp},\39{}$\&{int} %
\\{seq}${},\39{}$\&{long} \&{long} \\{phase2probes}${},\39{}$\&{moveseq} %
\\{sol})\1\1\2\2\6
${}\{{}$\1\6
\\{get\_global\_lock}(\,);\6
${}\\{solved}\PP;{}$\6
\&{if} ((\&{int}) \\{sol}${}.\\{size}(\,)>\\{target\_length}\W\\{target%
\_length}){}$\1\5
${}\\{missed\_target}\PP;{}$\2\6
\&{if} ${}(\\{seq}\E\\{next\_sequence}){}$\5
${}\{{}$\1\6
${}\\{display}(\\{cp},\39\\{seq},\39\\{phase2probes},\39\\{sol});{}$\6
${}\\{next\_sequence}\PP;{}$\6
\&{while} ${}(\\{queue}.\\{find}(\\{next\_sequence})\I\\{queue}.\\{end}(\,)){}$%
\5
${}\{{}$\1\6
\&{solution} ${}{\AND}\|s\K\\{queue}[\\{next\_sequence}];{}$\7
${}\\{display}(\|s.\\{cp},\39\|s.\\{seq},\39\|s.\\{phase2probes},\39\|s.%
\\{moves});{}$\6
${}\\{queue}.\\{erase}(\\{next\_sequence});{}$\6
${}\\{next\_sequence}\PP;{}$\6
\4${}\}{}$\2\6
\4${}\}{}$\2\6
\&{else}\5
${}\{{}$\1\6
${}\\{queue}[\\{seq}]\K\&{solution}(\\{cp},\39\\{seq},\39\\{phase2probes},\39%
\\{sol});{}$\6
\4${}\}{}$\2\6
\\{release\_global\_lock}(\,);\6
\4${}\}{}$\2\par
\fi

\M{30}When we are all done, we print this summary.

\Y\B\4\X30:Print summary\X${}\E{}$\6
\&{if} (\\{missed\_target})\1\5
${}\\{cout}\LL\.{"WARNING:\ \ missed\ ta}\)\.{rget\ on\ "}\LL\\{missed\_target}%
\LL\.{"\ sequences."}\LL\\{endl};{}$\2\6
\\{phase1prune}\DC\\{check\_integrity}(\,);\6
\\{phase2prune}\DC\\{check\_integrity}(\,);\6
${}\\{cout}\LL\.{"Solved\ "}\LL\\{solved}\LL\.{"\ sequences\ in\ "}\LL%
\\{duration}(\,)\LL\.{"\ seconds\ with\ "}\LL\\{phase2total}\LL\.{"\ probes."}%
\LL\\{endl};{}$\6
${}\\{cout}\LL\.{"Completed\ in\ "}\LL(\\{walltime}(\,)-\\{progstart})\LL%
\\{endl};{}$\6
\\{exit}(\T{0});\par
\U1.\fi

\M{31}Each thread has a routine that gets the next sequence to solve,
reports its solution, and moves on to the next, until there are
no more positions to solve.  This is that routine.

\Y\B\4\X13:Solver methods\X${}\mathrel+\E{}$\6
\&{void} \\{dowork}(\,)\1\1\2\2\6
${}\{{}$\1\6
\&{cubepos} \\{cp};\6
\&{int} \\{seq};\7
\&{while} (\T{1})\5
${}\{{}$\1\6
${}\\{seq}\K\\{getwork}(\\{cp});{}$\6
\&{if} ${}(\\{seq}\Z\T{0}){}$\1\5
\&{return};\2\6
${}\\{solve}(\\{seq},\39\\{cp});{}$\6
\4${}\}{}$\2\6
\4${}\}{}$\2\par
\fi

\M{32}The getwork routine grabs the global lock, gets another input
line, parses it, and returns the position and the sequence.

\Y\B\4\X10:Utility functions\X${}\mathrel+\E{}$\6
\&{int} \\{getwork}(\&{cubepos} ${}{\AND}\\{cp}){}$\1\1\2\2\6
${}\{{}$\1\6
\&{static} \&{int} \\{input\_seq}${}\K\T{1};{}$\6
\&{const} \&{int} \.{BUFSIZE}${}\K\T{1000};{}$\6
\&{char} ${}\\{buf}[\.{BUFSIZE}+\T{1}];{}$\7
\\{get\_global\_lock}(\,);\6
\&{if} ${}(\\{fgets}(\\{buf},\39\.{BUFSIZE},\39\\{stdin})\E\T{0}){}$\5
${}\{{}$\1\6
\\{release\_global\_lock}(\,);\6
\&{return} ${}{-}\T{1};{}$\6
\4${}\}{}$\2\6
\&{if} ${}(\\{cp}.\\{parse\_Singmaster}(\\{buf})\I\T{0}){}$\5
${}\{{}$\1\6
${}\\{cp}\K\\{identity\_cube};{}$\7
\&{const} \&{char} ${}{*}\|p\K\\{buf};{}$\6
\&{moveseq} \\{ms}${}\K\&{cubepos}\DC\\{parse\_moveseq}(\|p);{}$\7
\&{if} ${}({*}\|p){}$\1\6
\&{error} (\.{"!\ could\ not\ parse\ p}\)\.{osition"})\1\5
;\2\2\6
\&{for} (\&{unsigned} \&{int} \|i${}\K\T{0};{}$ ${}\|i<\\{ms}.\\{size}(\,);{}$
${}\|i\PP){}$\1\5
${}\\{cp}.\\{move}(\\{ms}[\|i]);{}$\2\6
\4${}\}{}$\2\7
\&{int} \|r${}\K\\{input\_seq}\PP;{}$\7
\\{release\_global\_lock}(\,);\6
\&{return} \|r;\6
\4${}\}{}$\2\par
\fi

\M{33}Pthreads wants a function, not a method, so we define this routine
to help it out.

\Y\B\4\X13:Solver methods\X${}\mathrel+\E{}$\6
\&{static} \&{void} ${}{*}{}$\\{worker}(\&{void} ${}{*}\|s){}$\1\1\2\2\6
${}\{{}$\1\6
\&{twophasesolver} ${}{*}\\{solv}\K{}$(\&{twophasesolver} ${}{*}){}$ \|s;\7
${}\\{solv}\MG\\{dowork}(\,);{}$\6
\&{return} \T{0};\6
\4${}\}{}$\2\par
\fi

\M{34}Our main body spawns the number of threads requested by the user
and waits for them to finish.  As a minor optimization, we use
the main thread for the thread zero work.

\Y\B\4\X34:Handle the work\X${}\E{}$\6
\\{pthread\_t}\\{p\_thread}[\.{MAX\_THREADS}];\6
\&{for} (\&{int} \\{ti}${}\K\T{1};{}$ ${}\\{ti}<\\{numthreads};{}$ ${}\\{ti}%
\PP){}$\1\5
${}\\{pthread\_create}({\AND}(\\{p\_thread}[\\{ti}]),\39\NULL,\39%
\&{twophasesolver}\DC\\{worker},\39\\{solvers}+\\{ti});{}$\2\6
${}\\{solvers}[\T{0}].\\{dowork}(\,);{}$\6
\&{for} (\&{int} \\{ti}${}\K\T{1};{}$ ${}\\{ti}<\\{numthreads};{}$ ${}\\{ti}%
\PP){}$\1\5
${}\\{pthread\_join}(\\{p\_thread}[\\{ti}],\39\T{0}){}$;\2\par

\U1.\fi


\inx
\fin
\con
